\documentclass[10pt,a4paper]{article}
\usepackage[utf8]{inputenc}
\usepackage{amsmath}
\usepackage{amsfonts}
\usepackage{amssymb}
\begin{document}

%Proposal follows a well-organized structure and would be readily understood by its intended audience. Each section is written in a clear, concise and specific manner. Few grammatical and spelling mistakes are present. All resources used and referenced are properly cited.

\section{Domain Background}
%The project's domain background — the field of research where the project is derived;

%Student briefly details background information of the domain from which the project is proposed. Historical information %relevant to the project should be included. It should be clear how or why a problem in the domain can or should be solved. %Related academic research should be appropriately cited. A discussion of the student's personal motivation for %investigating a particular problem in the domain is encouraged but not required.


\section{Problem Statement}
%A problem statement — a problem being investigated for which a solution will be defined;

% Student clearly describes the problem that is to be solved. The problem is well defined and has at least one relevant potential solution. Additionally, the problem is quantifiable, measurable, and replicable.

\section{Datasets and Inputs}
%The datasets and inputs — data or inputs being used for the problem;

% The dataset(s) and/or input(s) to be used in the project are thoroughly described. Information such as how the dataset or input is (was) obtained, and the characteristics of the dataset or input, should be included. It should be clear how the dataset(s) or input(s) will be used in the project and whether their use is appropriate given the context of the problem.

\section{Solution Statement}
%A solution statement — a the solution proposed for the problem given;

% Student clearly describes a solution to the problem. The solution is applicable to the project domain and appropriate for the dataset(s) or input(s) given. Additionally, the solution is quantifiable, measurable, and replicable.


\section{Benchmark Model}
%A benchmark model — some simple or historical model or result to compare the defined solution to;

% A benchmark model is provided that relates to the domain, problem statement, and intended solution. Ideally, the student's benchmark model provides context for existing methods or known information in the domain and problem given, which can then be objectively compared to the student's solution. The benchmark model is clearly defined and measurable.

\section{Evaluation Metrics}
%A set of evaluation metrics — functional representations for how the solution can be measured;

% Student proposes at least one evaluation metric that can be used to quantify the performance of both the benchmark model and the solution model presented. The evaluation metric(s) proposed are appropriate given the context of the data, the problem statement, and the intended solution.



\section{Project Design}
%An outline of the project design — how the solution will be developed and results obtained.

%Student summarizes a theoretical workflow for approaching a solution given the problem. Discussion is made as to what strategies may be employed, what analysis of the data might be required, or which algorithms will be considered. The workflow and discussion provided align with the qualities of the project. Small visualizations, pseudocode, or diagrams are encouraged but not required.

Data preprocessing\\
Splitting the data\\
model



\begin{thebibliography}{9}
\bibitem{latexcompanion} 
Michel Goossens, Frank Mittelbach, and Alexander Samarin. 
\textit{The \LaTeX\ Companion}. 
Addison-Wesley, Reading, Massachusetts, 1993.
 
\bibitem{einstein} 
Albert Einstein. 
\textit{Zur Elektrodynamik bewegter K{\"o}rper}. (German) 
[\textit{On the electrodynamics of moving bodies}]. 
Annalen der Physik, 322(10):891–921, 1905.
 
\bibitem{knuthwebsite} 
Knuth: Computers and Typesetting,
\\\texttt{http://www-cs-faculty.stanford.edu/\~{}uno/abcde.html}
\end{thebibliography}


\end{document}