\documentclass[10pt,a4paper]{article}
\usepackage[utf8]{inputenc}
\usepackage{amsmath}
\usepackage{amsfonts}
\usepackage{amssymb}
\usepackage{graphicx}
\usepackage{hyperref}
\usepackage{caption}
\usepackage{subcaption}

\usepackage{listings}
\usepackage{color}

\definecolor{dkgreen}{rgb}{0,0.6,0}
\definecolor{gray}{rgb}{0.5,0.5,0.5}
\definecolor{mauve}{rgb}{0.58,0,0.82}

\lstset{frame=tb,
  language=Python,
  aboveskip=3mm,
  belowskip=3mm,
  showstringspaces=false,
  columns=flexible,
  basicstyle={\small\ttfamily},
  numbers=none,
  numberstyle=\tiny\color{gray},
  keywordstyle=\color{blue},
  commentstyle=\color{dkgreen},
  stringstyle=\color{mauve},
  breaklines=true,
  breakatwhitespace=true,
  tabsize=3
}


\begin{document}

%Proposal follows a well-organized structure and would be readily understood by its intended audience. Each section is written in a clear, concise and specific manner. Few grammatical and spelling mistakes are present. All resources used and referenced are properly cited.

\begin{titlepage}
	\centering
	\vspace{1cm}
	{\scshape\Large Udacity 2018: Capstone Project \par}
	\vspace{1.5cm}
	{\huge\bfseries Forecasting time series with machine learning\par}
	\vspace{1.5cm}
	{\large\bfseries with applications to currency exchange rates \par}
	\vspace{2cm}
	{\Large Author: Oscar Javier Hernandez\par}
	\vfill

% Bottom of the page
	{\large \today\par}
\end{titlepage}

\section{Definition}
%(approx. 1-2 pages)
\subsection{Project Overview}\label{sec: overview}
%In this section, look to provide a high-level overview of the project in layman’s terms. Questions to ask yourself when writing this section:
%
%Has an overview of the project been provided, such as the problem domain, project origin, and related datasets or input data?
%Has enough background information been given so that an uninformed reader would understand the problem domain and following problem statement?
A set of data that is indexed by time is known as a time series. They appear in many different fields, such as statistics, physics, finance, economics, biology, or even business \cite{Adhikari_2013}. Because of their wide applicability, it is important to generate accurate forecasts of time series data. These forecasts are generated using specific mathematical models or algorithms which are trained on a subset of the past values of a given time series. For the purpose of simplifying future discussions, we will adopt the following notation for a time series, denoted $X(t)$ or $X_t$, as
\begin{equation}
\lbrace X(t); t=0,1,... \rbrace.
\end{equation}
Where $t$ denotes the time-index of the series. One of the simplest models for a time series is the ARIMA (Auto regressive integrated moving average) model. This model is denoted as ARIMA$(p,d,q)$, and assumes that the time series $X_t$ has the form
\begin{equation}
X_{t} = \mu +\epsilon_t+ \sum_{i=1}^p \phi_i L^i \left[ (1-L)^d \right] X_{t-i} + \sum_{j=1}^q \theta_j \epsilon_{t-j},
\end{equation}  
where $\lbrace \phi_i | i=1,...,p \rbrace$, $\lbrace \theta_i | i=1,...,q \rbrace$ are model parameters and $L$ is the lag operator defined as $L X_t = X_{t-1} $. The parameter $p$, is known as the auto-regressive (${\rm AR}$) order, $d$ is the differencing order, and $q$ is the moving average parameter (MA). 

The term $\epsilon_{t}$ denotes the error terms, assumed to be independent, identically distributed random variables sampled from a zero-mean, normal distribution. The value $\mu$ denotes the average of this model. ARIMA models can be applied to make forecasts of stationary time series ( defined as a time series whose mean, variance and auto correlation does not change over time), or to a time series that can be transformed into a stationary time series. However, there are other state-of-the-art machine learning methods that can be used to model time series methods. Which will the main goal of this project.

One important type of financial time series is the exchange rate between different currencies (Fig.~\ref{fig:EURUSD example}). An exchange rate, is the rate at which one currency will be exchanged for another. There are many factors that can influence this rate, such as balance of payments, interest rate levels, inflation levels and other economical factors which are beyond the scope of this project \cite{Patel_2014}.


\begin{figure}[h]
\begin{center}
\includegraphics[scale=0.4]{EURO_USD_exchange_rate.pdf}
\caption{The exchange rate from EUR to USD from Jan 2016 to Jul 2016.}
\label{fig:EURUSD example}
\centering
\end{center}
\end{figure}


\subsection{Project Statement}
%In this section, you will want to clearly define the problem that you are trying to solve, including the strategy (outline of tasks) you will use to achieve the desired solution. You should also thoroughly discuss what the intended solution will be for this problem. Questions to ask yourself when writing this section:
%
%Is the problem statement clearly defined? Will the reader understand what you are expecting to solve?
%Have you thoroughly discussed how you will attempt to solve the problem?
%Is an anticipated solution clearly defined? Will the reader understand what results you are looking for?

The main objective of this project will be to use classical and more recent machine learning techiques to make forecasts of different time series with the goal of applying the best methods to predict currency exchange rates. The simplest model that we will use is the ARIMA model, defined in the previous section as the baseline model, along with a linear regression model. The ARIMA model has been shown to be adequate in estimating the exchange rates of certain currencies Ref.~\cite{Mong_2016}. We will then use different neural networks architectures such as the feed forward and recurrent networks, as in Ref.~\cite{Chaudhuri_2016,Oancea_2014,Pant_2018}, to make predictions of time series and use our baseline models and root mean square differences to quantify and compare the performance of our different architectures.


\subsection{Metrics}
%In this section, you will need to clearly define the metrics or calculations you will use to measure performance of a model or result in your project. These calculations and metrics should be justified based on the characteristics of the problem and problem domain. Questions to ask yourself when writing this section:
%
%Are the metrics you’ve chosen to measure the performance of your models clearly discussed and defined?
%Have you provided reasonable justification for the metrics chosen based on the problem and solution?


There are several metrics that we can use to evaluate the predictions of our models Ref.~\cite{Adhikari_2013}, however, for our project we will focus on three commonly used metrics. We will first define some terminology, the forecast error, $e_t$, is given by
\begin{equation}
e_{t} = X_t - F_t,
\end{equation}
where $X_t$ is the value of the time series at time step $t$, and $F_t$ is the forecasted value at the same time step.

 The three metrics that we will use for our project are
\begin{enumerate}

\item  The mean square error (MSE)
\begin{itemize}
\item $\text{MSE} = \frac{1}{N}\sum\limits_{t=1}^{N} e^2_{t} $
\end{itemize}

\item  The root mean squared error (RMSE)
\begin{itemize}
\item $\text{RMSE} = \sqrt{\frac{1}{N}\sum\limits_{t=1}^{N} e^2_{t}} $
\end{itemize}

\item The Akaike information criterion (AIC)
\begin{itemize}
\item $\text{AIC} = 2k - \text{Ln}(\hat{L})$
\end{itemize}
where $k$ is the number of parameters in the model and $\hat{L}$ is the maximum value of the likelyhood function for the model.

\end{enumerate}
\noindent
In these three cases, the smaller the value of the RMSE, MSE, and AIC then the better the overall model.


\newpage
\section{Analysis}
%(approx. 2-4 pages)

\subsection{Data Exploration}
%In this section, you will be expected to analyze the data you are using for the problem. This data can either be in the form of a dataset (or datasets), input data (or input files), or even an environment. The type of data should be thoroughly described and, if possible, have basic statistics and information presented (such as discussion of input features or defining characteristics about the input or environment). Any abnormalities or interesting qualities about the data that may need to be addressed have been identified (such as features that need to be transformed or the possibility of outliers). Questions to ask yourself when writing this section:
%
%If a dataset is present for this problem, have you thoroughly discussed certain features about the dataset? Has a data sample been provided to the reader?
%If a dataset is present for this problem, are statistics about the dataset calculated and reported? Have any relevant results from this calculation been discussed?
%If a dataset is not present for this problem, has discussion been made about the input space or input data for your problem?
%Are there any abnormalities or characteristics about the input space or dataset that need to be addressed? (categorical variables, missing values, outliers, etc.)
For our project, we focused on three time series data sets. {\bf i}.~The international airline passenger data set, containing the total number of airline passengers in thousands from Jan 1949 until Dec 1960; {\bf ii}.~The sunspot data set, showing the monthly number of sunspots from 1705-1989; {\bf iii}.~The EUR to USD exchange rate from Jan 2016- July 2016. Datasets {\bf i}.~ and {\bf ii}.~ both consist of two columns, one for the time steps $t$ and the valued being measured $X_t$. 

\begin{table}[h]
\centering
\begin{tabular}{ c c }
Month $(t)$ & Int. airline passengers $(X_t)$ \\ \hline
 1949-01 & 112 \\ 
 1949-02 & 118  \\  
 1949-03 & 132   \\
 \vdots & \vdots 
\end{tabular}
\caption{The format of datasets {\bf i}.~and {\bf ii}.}
\label{table: sample format of dataset i and ii}
\end{table}

The third data set, {\bf iii}.~contains several columns related to the exchange rate market. The columns are Time, the Opening rate, the maximum value of the rate , the lowest value , the closing exchange rate value, and the total volume of the exchange market.

\begin{table}[h]
\centering
\begin{tabular}{ c c c c c c }
            Time &     Open &    High &    Low &   Close   &  Volume \\ \hline
2010-01-01 00:00 &  1.43283 & 1.43293 & 1.43224 & 1.43293 & 608600007.1 \\
2010-01-01 00:15 &  1.43285 & 1.43295 & 1.43229 & 1.43275 & 535600003.2 \\
2010-01-01 00:30 &  1.43280 & 1.43303 & 1.43239 & 1.43281 &  436299999.2 \\
\vdots  &  \vdots  & \vdots  & \vdots  & \vdots  &  \vdots 
\end{tabular}
\caption{A sample of the format of dataset {\bf iii}.}
\label{table: sample format of dataset iii}
\end{table}

Here we list some of the statistics of the dataset.

\begin{table}[h]
\centering
\begin{tabular}{c | c | c | c}
     &    Data set {\bf i.} & Data set {\bf ii.} &  Data set {\bf iii.} \\ \hline
count &  144 & 309 & 245441  \\
mean &  280.298611 & 49.752104 & 1.268375  \\
std  &  119.966317 & 40.452595 & 0.112992  \\
min  &  104.000000 & 0.0000000 & 1.035580 \\
25$\%$  &  180.000000 & 16.000000 & 1.135360 \\
50$\%$  &  265.500000 & 40.000000 & 1.302350  \\
75$\%$  &  360.500000 & 69.800000 & 1.356560 \\
max  &  622.000000 &  190.200000 & 1.493240 
\end{tabular}
\caption{Descriptive statistics of the data sets {\bf i}, {\bf ii}.~and {\bf iii}.}
\label{table: descriptive statistics of datasets}
\end{table}

\newpage
\subsection{Exploratory Visualization}
%In this section, you will need to provide some form of visualization that summarizes or extracts a relevant characteristic or feature about the data. The visualization should adequately support the data being used. Discuss why this visualization was chosen and how it is relevant. Questions to ask yourself when writing this section:
%
%Have you visualized a relevant characteristic or feature about the dataset or input data?
%Is the visualization thoroughly analyzed and discussed?
%If a plot is provided, are the axes, title, and datum clearly defined?
%
Here we produce visualizations of the three data sets. 


\begin{figure}[h]
\centering
\begin{subfigure}{.5\textwidth}
  \centering
  \includegraphics[scale=0.41]{Airline_Passengers.pdf}
  \caption{The number of airline passengers from 1949-1961.}
  \label{fig:Airline example}
\end{subfigure}%
\begin{subfigure}{.5\textwidth}
  \centering
  \includegraphics[scale=0.41]{Sunspot.pdf}
 \caption{The number of observed sunspots from 1705-1989. }
  \label{fig:Sunspot example}
\end{subfigure}
\caption{A figure with two subfigures}
\label{fig:test}
\end{figure}


The airline data set in Fig.~\ref{fig:Airline example} shows two interesting patterns, the general increasing number of airline passengers over time, along with a seasonal yearly pattern. The sunspot data set in Fig.~\ref{fig:Sunspot example} shows a cyclical pattern that repeats about every 11-years.

\begin{figure}[h]
\begin{center}
\includegraphics[scale=0.4]{EURO_USD_exchange_rate.pdf}
\caption{The exchange rate from EUR to USD from Jan 2016 to Jul 2016.}
\label{fig:EURUSD example2}
\centering
\end{center}
\end{figure}

In Fig.~\ref{fig:EURUSD example2} we plot the Euro to USD closing price from Jan 2016 to July 2016. There does not seem to be any apparent seasonal or cyclical patterns in this data sample.


\subsection{Algorithms and Techniques}
%In this section, you will need to discuss the algorithms and techniques you intend to use for solving the problem. You should justify the use of each one based on the characteristics of the problem and the problem domain. Questions to ask yourself when writing this section:
%
%Are the algorithms you will use, including any default variables/parameters in the project clearly defined?
%Are the techniques to be used thoroughly discussed and justified?
%Is it made clear how the input data or datasets will be handled by the algorithms and techniques chosen?
For this project, we will use the ARIMA model as a benchmark model as described in Sec.~\ref{sec: overview} and implemented as in Sec.~\ref{section: Fitting ARIMA}. ARIMA models have been used to predict foreign exchange rates \cite{Mong_2016} and are a classical method for time series forecasts Ref.~\cite{Adhikari_2013}. For these reasons, we use the ARIMA model as a benchmark.


Neural networks for time series models have been explored in the literature  \cite{Adhikari_2013,Oancea_2014,Chaudhuri_2016} and have shown good predictive abilities. This motivates us to use explore neural networks for time series modeling. The two neural networks that we will use is the feed-forward neural network with 4 nodes along with a long-short-term memory network with 4 LSTM units. Fig.~\ref{fig:FFNN architecture}, Fig.~\ref{fig:LSTM architecture}

\begin{figure}[h]
\begin{center}
\includegraphics[scale=0.4]{Neural_net_schematic_FFNN.pdf}
\caption{The chosen feed-forward neural network architecture.}
\label{fig:FFNN architecture}
\centering
\end{center}
\end{figure}


\begin{figure}[h]
\begin{center}
\includegraphics[scale=0.4]{Neural_net_schematic_LSTM.pdf}
\caption{The chosen LSTM neural network architecture.}
\label{fig:LSTM architecture}
\centering
\end{center}
\end{figure}


In Sec.~\ref{section: Fitting NN}, we will discus how these architectures were implemented in python.


\newpage
\subsection{Benchmark}

%In this section, you will need to provide a clearly defined benchmark result or threshold for comparing across performances obtained by your solution. The reasoning behind the benchmark (in the case where it is not an established result) should be discussed. Questions to ask yourself when writing this section:
%
%Has some result or value been provided that acts as a benchmark for measuring performance?
%Is it clear how this result or value was obtained (whether by data or by hypothesis)?
%

The ARIMA model and seasonal ARIMA model were fit with optimized grid-search parameters, as described in Sec.~\ref{section: Fitting ARIMA}, with optimized parameters given in Table \ref{table: optimal fitting parameters ARIMA}. In Fig.~\ref{fig:Airline ARIMA forecast}, Fig.~\ref{fig:Sunspot ARIMA forecast} and Fig.~\ref{fig:EURUSD ARIMA forecast} the curve in blue is the training data used for fitting, the green line in the testing data and the red lines show the average value of the ARIMA forecast while the red shaded area in the 95$\%$ confidence region of the prediction.

\begin{figure}[h]
\centering
\begin{subfigure}{.5\textwidth}
  \centering
  \includegraphics[scale=0.41]{Airline_Passengers_Forecast.pdf}
  \caption{SARIMA results}
  \label{fig:Airline ARIMA forecast}
\end{subfigure}%
\begin{subfigure}{.5\textwidth}
  \centering
  \includegraphics[scale=0.41]{Sunspot_Forecast.pdf}
\caption{ARIMA results}
  \label{fig:Sunspot ARIMA forecast}
\end{subfigure}
\caption{A figure with two subfigures}
\label{fig: Airline and SUNSPOT ARIMA}
\end{figure}

In Fiq.~\ref{fig:Airline ARIMA forecast}, we see that the SARIMA model was able to correctly find the overall trend and as time increases, the forecast uncertainty also increases as would be expected. Fig.~\ref{fig:Sunspot ARIMA forecast} which used a ARIMA model, did not seem to to detect the correct trend and produces an uncertainty band that does not change over time. However, the height of the band does overlap with the training data. 

\begin{figure}[h]
\begin{center}
\includegraphics[scale=0.4]{EURO_USD_exchange_rate_with_Forecast.pdf}
\caption{The forecasted exchange rate from EUR to USD from Jan 2016 to Jul 2016.}
\label{fig:EURUSD ARIMA forecast}
\centering
\end{center}
\end{figure}

In Fig.~\ref{fig:EURUSD ARIMA forecast} above, we see that the ARIMA model did not do a very good job at reproducing the data and it predicts a decreasing trend over time. The uncertainty band is very large in this case and encompasses a wide scale. In Table \ref{table: RMS values of benchmark models}, we provide the MSE and RMSE values of the optimized ARIMA and SARIMA models for the above data sets.

\begin{table}[h]
\centering
\begin{tabular}{lccc}
Data Set & Model & MSE & RMSE \\ \hline
{\bf i.} & SARIMA & 1878.63101438 & 43.3431772529 \\
{\bf ii.} & ARIMA & 902.466806466 & 30.0410853077 \\
{\bf iii.} & ARIMA & 8.67079188384e-05  &  0.00931170869596 \\ \hline
\end{tabular}
\caption{The RMS and RMSE values of the benchmark models.}
\label{table: RMS values of benchmark models}
\end{table}

%
\section{Methodology}
%(approx. 3-5 pages)
%
\subsection{Data Preprocessing}
%In this section, all of your preprocessing steps will need to be clearly documented, if any were necessary. From the previous section, any of the abnormalities or characteristics that you identified about the dataset will be addressed and corrected here. Questions to ask yourself when writing this section:
%
%If the algorithms chosen require preprocessing steps like feature selection or feature transformations, have they been properly documented?
%Based on the Data Exploration section, if there were abnormalities or characteristics that needed to be addressed, have they been properly corrected?
%If no preprocessing is needed, has it been made clear why?
%Implementation
%In this section, the process for which metrics, algorithms, and techniques that you implemented for the given data will need to be clearly documented. It should be abundantly clear how the implementation was carried out, and discussion should be made regarding any complications that occurred during this process. Questions to ask yourself when writing this section:
%
%Is it made clear how the algorithms and techniques were implemented with the given datasets or input data?
%Were there any complications with the original metrics or techniques that required changing prior to acquiring a solution?
%Was there any part of the coding process (e.g., writing complicated functions) that should be documented?
%Refinement
%In this section, you will need to discuss the process of improvement you made upon the algorithms and techniques you used in your implementation. For example, adjusting parameters for certain models to acquire improved solutions would fall under the refinement category. Your initial and final solutions should be reported, as well as any significant intermediate results as necessary. Questions to ask yourself when writing this section:
%
%Has an initial solution been found and clearly reported?
%Is the process of improvement clearly documented, such as what techniques were used?
%Are intermediate and final solutions clearly reported as the process is improved?

The forecasting models that we used in this work only require the values ($t,X_t$), since the data sets that we used do not contain any missing values, our data sets did not require any cleanup. However, it was necessary to scale the data, from $X_t \rightarrow \tilde{X}_t$ for the neural network data so that it was in the range $[0,1]$. This transformation was accomplished using the following scaling function
\begin{equation}
S(X_t) = \frac{ X_t - X_{\rm min}}{X_{\rm max} - X_{\rm min}},
\end{equation} 
where $X_{\rm max}, X_{\rm min}$ are the maximal and minimum values of the data sets, respectively. This is implemented in python as the following code snippet.
\begin{lstlisting}
def scale_array(y_vec):
    
    ymax = np.max(y_vec)
    ymin = np.min(y_vec)
    
    y_vec_scaled = np.zeros((len(y_vec),1))
    
    for k in range(0,len(y_vec)):
        y_vec_scaled[k][0] = (y_vec[k][0]-ymin)/(ymax-ymin)
    
    return y_vec_scaled,ymin,ymax
\end{lstlisting}

To invert the value from the scaling function, we use,
\begin{equation}
S^{-1}(\tilde{X}_t) = X_{\rm min} + \tilde{X}_t\cdot \left(X_{\rm max} - X_{\rm min} \right).
\end{equation}
The inversion of the transformation was carried out after the model made its predictions so that the predictions would be in the original range of the data set. This is implemented as follows,
\begin{lstlisting}
def invert_scaling(y_vec_scaled,ymin,ymax):
    
    y_vec = np.zeros(y_vec_scaled.shape)
        
    for k in range(0,len(y_vec_scaled)):
        y_vec[k] = ymin+(ymax-ymin)*y_vec_scaled[k]
    return y_vec
\end{lstlisting}

\subsection{ARIMA Model: Implementation}\label{section: Fitting ARIMA}

We used the ARIMA and the Seasonal ARIMA models from the python package \verb|statsmodels| (Ref.~\cite{Skipper_2010}). The ARIMA$(p,d,q)$ consists of the variables $(p,d,q)$, which are the non-seasonal AR order, differencing, and MA order, respectively. In order to fit this model to data, we use grid search from $(p,d,q)=(0,0,0) \rightarrow (p_{\rm Max},d_{\rm Max},q_{\rm Max}) $. To fit the ARIMA model we chose to either minimize the AIC, or the root mean square (RMS) value. The following code generates the grid that we will use to search the model space for the best fitting ARIMA model,
\begin{lstlisting}
	p = range(0,pMax+1)
	d = range(0,dMax+1)
	q = range(0,qMax+1)
	
	# This creates all combinations of p,q,d
	pdq = list(itertools.product(p, d, q))
\end{lstlisting}
For each item in the grid generated with the previous code snippet, we fit ARIMA$(p,d,q)$ model using the following code.
\begin{lstlisting}
for param in pdq:
		arma_mod = statsmodels.tsa.arima_model.ARIMA(train_data,order=param).fit()
\end{lstlisting}
Once the model is fit, the AIC or RMS value is computed and the set of parameters $(p,d,q)$ that generated the best score is taken to be the optimal model.\\


The seasonal ARIMA model that incorporates non-seasonal and seasonal factors as a multiplicative model. Therefore, we can schematically write
\begin{equation}
{\rm SARIMA} = {\rm ARIMA}(p,d,q) \times {\rm S}(P,D,Q,T),
\end{equation}
where the variables $(p,d,q)$ are the non-seasonal AR order, differencing, and MA order, respectively. The variables $(P,D,Q)$ denote the same variables, except for the seasonal component $S$. The variable $T$ in the $S$ component denotes the number of time steps for a single period. For example, if the time units are in Months, then $T$=6 indicates a half-year seasonal pattern. In order to fit the SARIMA$(p,q,d,P,D,Q,T)$ model to data, we use a gridsearch. The following code snippet generates a grid of $(p,d,q)=(0,0,0) \rightarrow (p_{\rm Max},d_{\rm Max},q_{\rm Max}) $ and $(P,D,Q)=(0,0,0) \rightarrow (p_{\rm Max},d_{\rm Max},q_{\rm Max},T) $, where we take $T$ as fixed.
\begin{lstlisting}
	p = range(0,pMax+1)
	d = range(0,dMax+1)
	q = range(0,qMax+1)
	t = [t]
	
	# This creates all combinations of p,q,d
	pdq = list(itertools.product(p, d, q))
	
	# This creates all combinations of the seasonal variables
	seasonal_pdq = [(x[0], x[1], x[2], x[3]) for x in list(itertools.product(p, d, q,t))]
\end{lstlisting}
For each combination, $(p,q,d,P,D,Q,T)$, we fit the SARIMA model using the code below.
\begin{lstlisting}
	for param in pdq:
		for param_seasonal in seasonal_pdq:
					mod = sm.tsa.statespace.SARIMAX(train_data,
                            					    order=param,
                            	  seasonal_order=param_seasonal,
                                     enforce_stationarity=False,
                                    enforce_invertibility=False)
				   results = mod.fit()
\end{lstlisting}


Using the above prescription for fitting the ARIMA and SARIMA models, the following table summarizes the results of the gridsearch.
\begin{table}[h]
\centering
\begin{tabular}{l | c | c | c}
 Data Set    &    Model  & Optimal $(p,d,q)$ &  Optimal $(P,S,Q,T)$ \\ \hline
{\bf i.} Airline Passengers & SARIMA & $(1, 1, 1)$ & $(1, 0, 0, 12)$ \\
{\bf ii.} Sunspots & ARIMA  & $(5,0,4)$ & - \\
{\bf iii.} EUR to USD rate & ARIMA & (2, 0, 2) & - 
\end{tabular}
\caption{The optimal fitting parameters of the ARIMA, or SARIMA model.}
\label{table: optimal fitting parameters ARIMA}
\end{table}


\subsection{Neural Network Model: Implementation}\label{section: Fitting NN}


The feed-forward neural network in \ref{fig:FFNN architecture} is implemented in Keras with the following code snippet which also fits the model using the \verb|mean_squared_error| metric and the \verb|adam| optimizer.
\begin{lstlisting}
model = Sequential()
model.add(Dense(4, input_dim=lags, activation='relu'))
model.add(Dense(1))
model.compile(loss='mean_squared_error', optimizer='adam')
history=model.fit(X_train, y_train, epochs=number_of_epochs, batch_size=2, verbose=1)
\end{lstlisting}
The following is a summary of this implemented architecture
\begin{lstlisting}
Layer (type)                 Output Shape              Param #   
=================================================================
dense_1 (Dense)              (None, 4)                 8         
_________________________________________________________________
dense_2 (Dense)              (None, 1)                 5         
=================================================================
Total params: 13
Trainable params: 13
Non-trainable params: 0
_________________________________________________________________
\end{lstlisting}

\newpage
Using the same optimizer and loss function as the feed-forward neural network the LSTM network is implemented as follows
\begin{lstlisting}
model = Sequential()
model.add(LSTM(4, input_dim=lags))
model.add(Dense(1))
model.compile(loss='mean_squared_error', optimizer='adam')
history=model.fit(X_train, y_train, epochs=number_of_epochs, batch_size=2, verbose=1)
\end{lstlisting}
Below, we summarize the implemented LSTM network,
\begin{lstlisting}
Layer (type)                 Output Shape              Param #   
=================================================================
lstm_1 (LSTM)                (None, 4)                 96        
_________________________________________________________________
dense_1 (Dense)              (None, 1)                 5         
=================================================================
Total params: 101
Trainable params: 101
Non-trainable params: 0
_________________________________________________________________
\end{lstlisting}


\section{Results}
%(approx. 2-3 pages)
%
\subsection{Model Evaluation and Validation}
%In this section, the final model and any supporting qualities should be evaluated in detail. It should be clear how the final model was derived and why this model was chosen. In addition, some type of analysis should be used to validate the robustness of this model and its solution, such as manipulating the input data or environment to see how the model’s solution is affected (this is called sensitivity analysis). Questions to ask yourself when writing this section:
%
%Is the final model reasonable and aligning with solution expectations? Are the final parameters of the model appropriate?
%Has the final model been tested with various inputs to evaluate whether the model generalizes well to unseen data?
%Is the model robust enough for the problem? Do small perturbations (changes) in training data or the input space greatly affect the results?
%Can results found from the model be trusted?
Having implemented the feed forward and LSTM networks in Keras as described in Sec.~\ref{section: Fitting NN} we fit the model and generate the time series predictions $\lbrace F_t \rbrace$ 

\begin{figure}[h]
\centering
\begin{subfigure}{.5\textwidth}
  \centering
  \includegraphics[scale=0.41]{Airline_Passengers_NN.pdf}
  \caption{NN results}
  \label{fig:sub1}
\end{subfigure}%
\begin{subfigure}{.5\textwidth}
  \centering
  \includegraphics[scale=0.41]{Airline_Passengers_LSTN.pdf}
  \caption{LSTM results}
  \label{fig:sub2}
\end{subfigure}
\caption{A figure with two subfigures}
\label{fig:test}
\end{figure}

\newpage
\begin{figure}[h]
\centering
\begin{subfigure}{.5\textwidth}
  \centering
  \includegraphics[scale=0.41]{Sunspot_Forecast_NN.pdf}
  \caption{NN results}
  \label{fig:sub1}
\end{subfigure}%
\begin{subfigure}{.5\textwidth}
  \centering
  \includegraphics[scale=0.41]{Sunspot_Forecast_LSTN.pdf}
  \caption{LSTM results}
  \label{fig:sub2}
\end{subfigure}
\caption{A figure with two subfigures}
\label{fig:test}
\end{figure}


\begin{figure}[h]
\centering
\begin{subfigure}{.5\textwidth}
  \centering
  \includegraphics[scale=0.41]{EURO_USD_exchange_rate_NN.pdf}
  \caption{NN results}
  \label{fig:sub1}
\end{subfigure}%
\begin{subfigure}{.5\textwidth}
  \centering
  \includegraphics[scale=0.41]{EURO_USD_exchange_rate_LSTN.pdf}
  \caption{LSTM results}
  \label{fig:sub2}
\end{subfigure}
\caption{A figure with two subfigures}
\label{fig:test}
\end{figure}



\begin{table}[h]
\centering
\begin{tabular}{cccc}
Data Set & Model & MSE & RMSE \\ \hline
{\bf i.} & SARIMA & 1878.63101438 & 43.3431772529 \\
{\bf i.} & FFNN & 604.999299329 & 24.5967335093 \\
{\bf i.} & LSTM & 567.812500312 & 23.8288165949 \\ \hline 
{\bf ii.} & ARIMA & 902.466806466 & 30.0410853077 \\
{\bf ii.} & FFNN & 101.171109089 & 10.058385014 \\
{\bf ii.} & LSTM & 119.392504984 & 10.926687741 \\ \hline
{\bf iii.} & ARIMA & 8.67079188384e-05  &  0.00931170869596 \\
{\bf iii.} & FFNN & 7.24275099e-08 & 0.000269123596 \\
{\bf iii.} & LSTM & 8.45390421e-08 & 0.000290755984 \\ \hline
\end{tabular}
\caption{The RMS and RMSE values of the Feed Forward (FFNN) and long-short term memory (LSTM) neural network model results.}
\label{table: RMS values of neural network models}
\end{table}

\newpage
\subsection{Justification}
%In this section, your model’s final solution and its results should be compared to the benchmark you established earlier in the project using some type of statistical analysis. You should also justify whether these results and the solution are significant enough to have solved the problem posed in the project. Questions to ask yourself when writing this section:
%
%Are the final results found stronger than the benchmark result reported earlier?
%Have you thoroughly analyzed and discussed the final solution?
%Is the final solution significant enough to have solved the problem?


\section{Conclusion}
%(approx. 1-2 pages)
%
\subsection{Free-Form Visualization}
%In this section, you will need to provide some form of visualization that emphasizes an important quality about the project. It is much more free-form, but should reasonably support a significant result or characteristic about the problem that you want to discuss. Questions to ask yourself when writing this section:
%
%Have you visualized a relevant or important quality about the problem, dataset, input data, or results?
%Is the visualization thoroughly analyzed and discussed?
%If a plot is provided, are the axes, title, and datum clearly defined?

\subsection{Reflection}
%In this section, you will summarize the entire end-to-end problem solution and discuss one or two particular aspects of the project you found interesting or difficult. You are expected to reflect on the project as a whole to show that you have a firm understanding of the entire process employed in your work. Questions to ask yourself when writing this section:
%
%Have you thoroughly summarized the entire process you used for this project?
%Were there any interesting aspects of the project?
%Were there any difficult aspects of the project?
%Does the final model and solution fit your expectations for the problem, and should it be used in a general setting to solve these types of problems?
Through this project, we have 

\subsection{Improvement}
%In this section, you will need to provide discussion as to how one aspect of the implementation you designed could be improved. As an example, consider ways your implementation can be made more general, and what would need to be modified. You do not need to make this improvement, but the potential solutions resulting from these changes are considered and compared/contrasted to your current solution. Questions to ask yourself when writing this section:
%
%Are there further improvements that could be made on the algorithms or techniques you used in this project?
%Were there algorithms or techniques you researched that you did not know how to implement, but would consider using if you knew how?
%If you used your final solution as the new benchmark, do you think an even better solution exists?

While we have obtained interesting results using the machine learning approach, in the future we would like to implement a generative artificial neural network.


%Before submitting, ask yourself. . .
%
%Does the project report you’ve written follow a well-organized structure similar to that of the project template?
%Is each section (particularly Analysis and Methodology) written in a clear, concise and specific fashion? Are there any ambiguous terms or phrases that need clarification?
%Would the intended audience of your project be able to understand your analysis, methods, and results?
%Have you properly proof-read your project report to assure there are minimal grammatical and spelling mistakes?
%Are all the resources used for this project correctly cited and referenced?
%Is the code that implements your solution easily readable and properly commented?
%Does the code execute without error and produce results similar to those reported?




\newpage
\begin{thebibliography}{9}

\bibitem{Adhikari_2013}
R. Adhikari and R. K. Agrawal, An Introductory Study on Time Series Modeling and Forecasting, 2013, [arXiv:1302.6613] \url{https://arxiv.org/abs/1302.6613}.

\bibitem{Mong_2016}
T. Mong and U. Ngan, Research Journal of Finance and Accounting www.iiste.org
ISSN 2222-1697 (Paper) ISSN 2222-2847 (Online)
Vol.7, No.12, 2016 \url{http://iiste.org/Journals/index.php/RJFA/article/viewFile/31511/32351}.

\bibitem{Patel_2014} P. J. Patel,  N. J. Patel and A. R. Patel, IJAIEM 3, 3, 2014. \url{http://www.ijaiem.org/volume3issue3/IJAIEM-2014-03-05-013.pdf}

% Neural networks for time Series
\bibitem{Oancea_2014} B. Oancea, S. Cristian Ciucu, Proceedings of the CKS 2013, [arXiv:1401.1333] \url{https://arxiv.org/abs/1401.1333}.

% Neural network time series  
\bibitem{Chaudhuri_2016} T. D. Chaudhuri and I. Ghosh,	Journal of Insurance and Financial Management, Vol. 1, Issue 5, PP. 92-123, 2016,  [arXiv:1607.02093] \url{https://arxiv.org/abs/1607.02093}.

% Tutorial for LSTM for time series 
\bibitem{Pant_2018} Pant, N. (2017, September 07). A Guide For Time Series Prediction Using Recurrent Neural Networks (LSTMs). Retrieved from \url{https://blog.statsbot.co/time-series-prediction-using-recurrent-neural-networks-lstms-807fa6ca7f}.


% Tutorial for Feed forward neural network
\bibitem{Acatay_2017} D. K. Acatay, (2017, Nov. 21) Part 6: Time Series Prediction with Neural Networks in Python. Retrieved from \url{http://dacatay.com/data-science/part-6-time-series-prediction-neural-networks-python/}

\bibitem{Vincent_2018} Vincent, T. (2018). ARIMA Time Series Data Forecasting and Visualization in Python | DigitalOcean. [online] digitalocean.com. Available at: \url{https://www.digitalocean.com/community/tutorials/a-guide-to-time-series-forecasting-with-arima-in-python-3} [Accessed 29 Jul. 2018].

\bibitem{Esteban_2017} C. Esteban, S. L. Hyland and G R\"{a}tsch, \url{https://github.com/ratschlab/RGAN} [arXiv:1706.02633]. 

\bibitem{Skipper_2010} S. Skipper and J. Perktold. ``Statsmodels: Econometric and statistical modeling with python." Proceedings of the 9th Python in Science Conference. 2010. \url{https://www.statsmodels.org/stable/index.html}



\end{thebibliography}
\end{document}